\documentclass[14pt]{extarticle}
\usepackage[utf8]{inputenc}
\usepackage{indentfirst}
\usepackage{mathptmx}
\usepackage[T1]{fontenc}
\usepackage[polish]{babel}
\usepackage{lmodern}
\usepackage{fancyhdr}
\usepackage{changepage}
\selectlanguage{polish}
\usepackage{amsfonts}
\usepackage{graphicx}
\usepackage[margin=0.5in]{geometry}
\addtolength{\topmargin}{.250in}

\pagestyle{fancy}
\lhead[E]{11. Więcej o relacjach}


\title{Relacja porządku - wright}
\author{Kamil Marynowski - nr albumu 71610 Grupa IAM1 - Informatyka inżynierskie }
\date{February 2021}

\def\changemargin#1#2{\list{}{\rightmargin#2\leftmargin#1}\item[]}
\let\endchangemargin=\endlist 

\begin{document}
\thispagestyle{empty}

% \begin{center}
%     {\fontsize{90}{10}\selectfont 11.} {\fontsize{28}{10}\selectfont  \space WIĘCEJ O RELACJACH}    
% \end{center}

\textbf{
    {\fontsize{80}{10}\selectfont 11.}
    \quad
    {\fontsize{20}{10}\selectfont WIĘCEJ O RELACJACH}    
}
\bigbreak
\bigbreak
\bigbreak
\bigbreak
\bigbreak
\bigbreak

\begin{changemargin}{160pt}{0pt} 
\qquad W tym rozdziale kontynuujemy badanie relacji rozpoczęte
dawno temu, w rodziale 3. Rozsądnie byłoby pokrótce go przej-
rzeć dla przypomnienia sobie terminologii. Pierwsze dwa pa-
ragrafy tego rodziału zajmują się relacjami, które porządkują
elementy danego zbioru. Zaczniemy od częściowych porządków
w przypadku ogólnym, a następnie zajmiemy się pewnymi 
szczególnymi relacjami porządku na zbiorach.
$S_{1} \times ... \times S_{n}$ i $\sum^{*}$. W paragrafie 11.3 omawiamy pojęcie złożenia dowolnych relacji i pokazujemy, jak stwierdzenia dotyczące relacji można wyrażać w języku macierzy. W ostatnim paragrafie zajmujemy się pojęciem najmniejszej relacji, spełniającej różnorodne własności i zawierającej daną relację \textit{R} w zbiorze \textit{S}. Opisana jest tam, w szczególności, najmniejsza relacja równoważności zawierająca \textit{R}.

\qquad Paragrafy 11.3 i 11.4 są niezależne od pierwszych dwóch paragrafów tego rodziału i mogą być studiowane osobno.
\end{changemargin} 


\section*{ \textsection~11.1. Zbiory częściowo uporządkowane}
\begin{changemargin}{160pt}{0pt} 
\qquad W tym paragrafie zajmiemy się zbiorami, których elementy
można ze sobą w pewien sposób porównywać. W typowej sytuacji jeden z elementów będziemy traktować jako mniejszy od drugiego bądź jako występujący przed nim w sensie pewnej kolejności.
\end{changemargin} 

\textbf{PRZYKŁAD 1} 
\begin{changemargin}{160pt}{0pt} 
\qquad (a) Przyzwyczajeni jesteśmy do porównywania liczb 
rzeczywistych.
Na przykład liczba 3 jest mniejsza od liczby 5, liczba -1 jest mniejsza od liczby 4 i liczba -1 jest większa od liczby -3. Porównujemy dwie liczby stwierdzając, która z nich jest większa, a która mniejsza.
\end{changemargin}

\begin{changemargin}{160pt}{0pt} 
\qquad (b) Jeśli elementy jakiegoś zbioru \textit{S} są wypisane z użyciem
indeksów ze zbioru $\mathbb{P}$ lub $\mathbb{N}$ w taki sposób, że różne elementy mają różne indeksy, to możemy porównać dwa elementy zbioru \textit{S} stwierdzając, który z nich ma mniejszy indeks. Różne sposoby indeksowania elementów \textit{S} prowadzą do różnych sposobów ustalenia kolejności elementów \textit{S}. Dany element raz może mieć najmniejszy indeks, innym razem zaś może być poprzedzony wieloma innymi elementami.
\end{changemargin}

\begin{changemargin}{160pt}{0pt} 
\qquad Zbiór, którego elementy można w ten sposób porównywać, nazywamy \textbf{uporządkowanym}, a strukturę, która niesie informację, jak jego elementy ze sobą porównywać, nazywamy \textbf{relacją porządku} w zbiorze \textit{S}. Aby móc powiedzieć cokolwiek pożytecznego o zbiorach uporządkowanych, musimy uczynić te definicje bardziej precyzyjnymi. Na początek wszakże zauważymy, że w przypadku wielu pojawiających się w sposób naturalny zbiorów wiemy, jak porównywać pewne elementy z innymi, ale jednocześnie mamy do czynienia z parami elementów nieporównywalnych.
\end{changemargin}

\textbf{PRZYKŁAD 2}
\begin{changemargin}{160pt}{0pt} 
\qquad (a) Gdybyśmy chcieli porównywać ze sobą marki samochodów, to moglibyśmy się, być może zgodzić co do tego, że marka \textit{RR} jest lepsza od marki \textit{H}, ponieważ jest od niej lepsza pod każdym względem. Moglibyśmy jendak nie być w stanie powiedzieć czy lepsza jest marka \textit{F}, czy marka \textit{C}, ponieważ każda z nich może pod pewnymi względami drugą przewyższać.

\qquad (b) Możemy się umówić, że dwie liczby ze zbioru 
$\{$1, 2, 3, ...,73$\}$ są porównywalne, jeśli jedna z nich jest dzielnikiem drugiej. Wówczas liczby 6 i 72, a także 6 i 3, są porównywalne. Natomiast liczby 6 i 8 nie są porównywalne, gdyż żadna z nich nie jest dzielnikiem drugiej.

\qquad (c) Możemy porównywać dwa pozdbiory danego zbioru \textit{S} (tzn. elementy zbioru $\mathcal{P}(\mathit{S}))$, jeśli jeden z nich jest zawarty w drugim. 
Jeśli zbiór \textit{S} ma więcej, niż jeden element to zawiera jakieś nieporównywalne podzbiory. Na przykład, jeśli $s_{1} \ne s_{2}$ oraz $s_{1}$ i $s_{2}$ należą do \textit{S}, to zbiory $\{s_{1}\}$ i $\{s_{2}\}$ są nieporównywalne.

\qquad (d) Możemy porównywać funckje punktowo. Jeśli, na przykład, funkcje \textit{f} i \textit{g} są określone na zbiorze \textit{S} i przyjmują wartości w $\{0, 1\}$, to moglibyśmy traktować funkcję \textit{f} jako mniejszą lub równą \textit{g}, gdy \textit{f}(\textit{s}) $\le$ \textit{g}(\textit{s}) dla każdego $\mathit{s} \in \mathit{S}$. To jest w gruncie rzeczy ten porządek, który nadaliśmy zbiorowi $\mathbb{B}^{n}$ w przykładzie 3 z \textsection 10.1. Jest on podobny do sposobu porównywania samochodów z punktu (a).

\qquad Zbiory, takie jak te z przykładu 2, z relacjami pozwalającymi porównywać ze sobą elementy, ale dopuszczającymi istnienie elementów nieporównywalnych nazywamy częściowo uporządkowanymi. Tworzą one ważną klasę, którą teraz zdefiniujemy w ścisły sposób.

\qquad Przypomnij sobie, że relacja \textit{R} w zbiorze \textit{S} jest to podzbiór
zbioru $\textit{S} \times \textit{S}$. \textbf{Częściowy porządek} w zbiorze \textit{S} to relacja \textit{R}, która jest zwrotna antysymetrzyczna i przechodnia. Jeśli będziemy pisać $\textit{x} \preceq \textit{y}$ zamiast $(\textit{x},\textit{y}) \in \textit{R}$, to warunki te stwierdzają, że częściowy porządek spełnia:
\begin{itemize}
    \item[] (Z) \quad $\mathit{s} \preceq \mathit{s}$ dla każdego $\mathit{s} w \mathit{S};$
    \item[] (AS) \quad  $\mathit{s} \preceq \mathit{t}$ i $\mathit{t} \preceq \mathit{s}$ implikują $s = t;$
    \item[] (P) \quad $\mathit{s} \preceq \mathit{t}$ i $\mathit{t} \preceq \mathit{u}$ implikują $\mathit{s} \preceq \mathit{u}$.
\end{itemize}
To są dokładnie te własności zwykłego porządku $\le$, które wyróżniliśmy w przykładzie 4 z \textsection ~3.1.


\qquad Jeśli $\preceq$ jest częściowym porządkiem w zbiorze \textit{S}, to parę $(\textit{S}, \preceq)$ nazywamy \textbf{zbiorem częściowo uporządkowanym}. 
Używamy symbolu ''$\preceq$''~jako ogólnego oznaczenia dla częściowego porządku.
Jeśli dla danego szczególnego częściowego porządku istnieje już oznaczenie, takie jak ''$\le$'' lub ''$\subseteq$'', to będziemy raczej używali właśnie jego zamiast ''$\preceq$''.

\qquad W przykładzie 2 chodziło o relacje ''jest nie tak dobry jak'', ''jest dzielnikiem'', ''jest podzbiorem'' i ''w żadnym punkcie nie jest większa''. Moglibyśmy równie dobrze rozważać relacje ''jest co najmniej tak dobry jak'', ''jest wielokrotnością'', ''zawiera'', i ''w każdym punkcie jest co najmniej równa'', ponieważ relacje te dostarczają tych samych informacji dotyczących porównywania elementów, co relacje przez nas wybrane. Każdy częsćiowy porządek w danym zbiorze wyznacza \textbf{relację odwrotną}, która wiąże \textit{x} z \textit{y} wtedy i tylko wtedy, gdy \textit{y} pozostaje w relacji wyjściowej z \textit{x}. Relacja odwrotna do częściowego porządku $\preceq$ jest na ogół oznaczana przez $\succeq$. Zatem \textit{x} $\succeq$ \textit{y} znaczy to samo, co \textit{y} $\preceq$ \textit{x}. Ta relacja odwrotna również jest częściowym porządkiem (ćwiczenie 7(a)). Jeśli spojrzymy na relację $\preceq$ w zbiorze \textit{S} jako na podzbiór \textit{R} zbioru \textit{S} $\times$ \textit{S}, to relacja $\succeq$
odpowiada relacji odwrotnej $\mathit{R^\gets}$, zdefiniowanej w \textsection ~3.1.

\qquad Mając dany częściowy porządek $\preceq$ w zbiorze $\textit{S}$, możemy zdefiniować jeszcze jedną relację, $\prec$, w $\mathit{S}$ w następujący sposób:

\begin{itemize}
\item[] $\mathit{x} \prec \mathit{y}$ wtedy i tylko wtedy, gdy $\mathit{x} \preceq \mathit{y}$ i $\mathit{x} \ne \mathit{y}$
\end{itemize}

Jeśli, na przykład, $\preceq$ oznacza inkluzję, to $\mathit{A} \prec \mathit{B}$ znaczy, że $\mathit{A}$ jest właściwym podzbiorem $\mathit{B}$, tzn. $\mathit{A} \subset \mathit{B}$. Relacja $\prec$ jest przeciwzwrotna i przechodnia:

\begin{itemize}
    \item[] (PZ) $\mathit{s} \prec \mathit{s}$ nie zachodzi dla żadnego $\mathit{s}$ w $\mathit{S}$;
    \item[] (P) $\mathit{s} \prec \mathit{t}$ i $\mathit{t} \prec \mathit{u}$ implikują $\mathit{s} \preceq \mathit{u}$.
\end{itemize}

Relację przeciwzwrotną i przechodnią nazywamy\newline \textbf{quasi-porządkiem}. Każdy częściowy porządek w zbiorze $\mathit{S}$ wyznacza pewien quasi-porządek i, na odwrót, jeśli $\prec$ jest quasi-porządkiem w $\mathit{S}$, to relacja $\preceq$ zdefiniowana formułą

\begin{itemize}
    \item[] $\mathit{x} \preceq \mathit{y}$ wtedy i tylko wtedy, gdy $\mathit{x} \prec \mathit{y}$ lub $\mathit{x} = \mathit{y}$
\end{itemize}

jest częściowym porządkiem w $\mathit{S}$ (ćwiczenie 7(b)). To, czy do porównywania elementów danego zbioru częściowo uporządkowanego wybierze się częściowy porządek, czy też związany z nim quasi-porządek, zależy od problemu, z którym mamy w danym momencie do czynienia. Będziemy na ogół używali częściowego porządku, ale jeśli tak będzie wygodnie, będziemy również używać na zmianę obu relacji.

\qquad Możliwe jest, przynajmniej teoretycznie, narysowanie diagramu, który pozwala jednym rzutem oka objąć całą relację porządku w skończonym zbiorze częściowo uporządkowanym. Mając dany częściowy porządek $\preceq$ w $\mathit{S}$, powiemy, że element $\mathit{t}$ \textbf{nakrywa} element $\mathit{s}$, gdy $\mathit{s} \prec \mathit{t}$ i nie ma w $\mathit{S}$ elementu $\mathit{u}$ takiego, że $\mathit{s} \prec \mathit{u} \prec \mathit{t}$. \textbf{Diagram Hassego} zbioru częściowo uporządkowanego $(\mathit{S},\preceq)$ jest to rysunek grafu skierowanego, którego wierzchołkami są elementy zbioru $\mathit{S}$ i w którym od wierzchołka $\mathit{t}$ do wierzchołka $\mathit{s}$ krawędź biegnie wtedy i tylko wtedy, gdy $\mathit{t}$ nakrywa $\mathit{s}$. Diagramy Hassego, podobnie jak drzewa z wyróżnionym korzeniem, są zazwyczaj rysowane z krawędziami skierowanymi w dół i bez strzałek.
\end{changemargin}

\newpage

\textbf{PRZYKŁAD 3} 
\begin{changemargin}{160pt}{0pt} 
\qquad (a) Niech $\mathit{S} = \{1,2,3,4,5,6\}$. Będziemy pisać $\mathit{m}|\mathit{n}$, gdy $\mathit{m}$ dzieli $\mathit{n}$, tzn. gdy $\mathit{n}$ jest całkowitą wielokrotnością $\mathit{m}$. Diagram z rysunku 11.1 jest diagramem Hassego zbioru częściowo uporządko-

\begin{figure}[h!]
\centering
\includegraphics[scale=1.25]{hassego.PNG}
\caption{}
\label{fig:universe}
\end{figure}

wanego $(\mathit{S},|)$. Między 1 i 6 nie ma krawędzi, ponieważ 6 nie nakrywa 1. Możemy wszakże wywnioskować z tego diagramu, że 1|6, ponieważ relacja | jest przechodnia i istnieje w diagramie łańcuch krawędzi odpowiadających temu, że 1|2 i 2|6. Podobnie możemy zobaczyć, że 1|4 biorąc pod uwagę drogę 1|2|4. Zauważ, że, ogólnie w przypadku relacji przechodnich stosować można takie łączone zapisy i nie prowadzi to do nieporozumień: $\mathit{x} \preceq \mathit{y} \preceq \mathit{z}$ znaczy to samo, co $\mathit{x} \preceq \mathit{y}$, $\mathit{y} \preceq \mathit{z}$ i $\mathit{x} \preceq \mathit{z}$.

\begin{figure}[h!]
\centering
\includegraphics[scale=1.25]{rys2.PNG}
\caption{}
\label{fig:universe}
\end{figure}

\qquad (b) Rozważmy zbiór potęgowy $\mathcal{P}(\{\mathit{a},\mathit{b}, \mathit{c}\})$ z częściowym porządkiem $\in$. Rysunek 2. przedstawia diagram Hassego zbioru $(P({a,b,c}),\subseteq)$. Zauważ, że linia łączaca $\{\mathit{a},\mathit{c}\}$ z $\{\mathit{a}\}$ przecina się z linią łączącą $\{\mathit{a},\mathit{b}\}$ z $\{\mathit{b}\}$, ale przecięcie to jest po prostu cechą tego rysunku i jest bez znaczenia z punktu widzenia własności przedstawianego częściowego porządku. W szczególności, przecięcie tych dwóch linii nie reprezentuje elementu naszego zbioru.

\begin{figure}[h!]
\centering
\includegraphics[scale=1.25]{rys3.PNG}
\caption{}
\label{fig:universe}
\end{figure}

\qquad (c) Diagram przedstawiony na rysunku 3. nie jest diagramem Hassego, ponieważ $\mathit{u}$ nie może nakrywać $\mathit{x}$, skoro $\mathit{u}$ nakrywa także $\mathit{y}$, a $\mathit{y}$ nakrywa $\mathit{x}$. Gdyby którakolwiek z krawędzi łączących $\mathit{u}, \mathit{x}$ i $\mathit{y}$ została usunięta, rysunek stałby się diagramem Hassego.

\begin{figure}[h!]
\centering
\includegraphics[scale=1.25]{rys4.PNG}
\caption{}
\label{fig:universe}
\end{figure}

\qquad (d) Diagramy przedstawione na rys. 4. są diagramami Hassego pewnych zbiorów częściowo uporządkowanych. Ich relacje porządku mogą być odczytane bezpośrednio z odpowiednich diagramów. Wszystkie elementy są w relacji same ze sobą. Ponadto:

\qquad Dla zbioru $\mathit{S}=\{a,b,c,d,e,f\}$ mamy $\mathit{a} \preceq \mathit{b}, \mathit{a} \preceq \mathit{c}, \mathit{a} \preceq \mathit{d}, \mathit{a} \preceq \mathit{e}, \mathit{a} \preceq \mathit{f}, \mathit{b} \preceq \mathit{e}, \mathit{b} \preceq \mathit{f}, \mathit{c} \preceq \mathit{f}$. Widzielśmy ten rysunek już wcześniej, w części (a) tego przykładu.

\qquad Dla zbioru $\mathit{T} = \{\mathit{x}, \mathit{y}, \mathit{z}, \mathit{w}\}$ mamy $\mathit{x} \preceq \mathit{y}, \mathit{x} \preceq \mathit{z}, \mathit{x} \preceq \mathit{w}, \mathit{y} \preceq \mathit{z}, \mathit{y} \preceq \mathit{w} i \mathit{z} \preceq \mathit{w}$, Taki właśnie rysunek otrzymalibyśmy dla zbioru wszystkich dzielników liczby 8, albo 27 lub 125 z relacją porządku |.

\qquad Dla zbioru $\mathit{U} = \{\mathit{A}, \mathit{B}, \mathit{C}, \mathit{D}, \mathit{E}\}$ mamy $\mathit{A} \preceq \mathit{B}, \mathit{A} \preceq \mathit{C}, \mathit{A} \preceq \mathit{D}, \mathit{A} \preceq \mathit{E}, \mathit{B} \preceq \mathit{E}, \mathit{C} \preceq \mathit{E} i \mathit{D} \preceq \mathit{E}$. Ten rysunek jest diagramem Hassego zbioru częściowo uporządkowanego składającego się ze zbiorów $\{1\}$, $\{1,2\}$, $\{1,3\}$, $\{1,4\}$, $\{1,2,3,4\}$ z inkluzją jako relacją porządku.

\qquad (e) Relacja <, którą zdefiniowaliśmy w algebrze Boole'a w \textsection ~10.1 jest quasi-porządkiem. Atomy są to dokładnie te elementy algebry, które nakrywają element 0. Zbiór częściowo uporządkowany $(\mathcal{P}(\{a,b,c\}), C)$, rozważany powyżej części (b),
stanowi przykład algebry Boole'a traktowanej jako zbiór częsciowo uporządkowany.\newline

\qquad Ogólnie, jeśli dany jest diagram Hassego pewnego zbioru częsciowo uporządkowanego, to widzimy, że $\mathit{s} \preceq \mathit{t}$, gdy bądź $\mathit{s} = \mathit{t}$, bądź istnieje (biegnąca w dół) droga od $\mathit{t}$ do $\mathit{s}$. Pamiętając, że porządek jest zwrotny i przechodni dowiemy się o nim wszystkiego na podstawie informacji dotyczących wzajemnego nakrywania elementów.

\qquad Fakt, że każdy skończony zbiór częściowo uporządkowany ma diagram Hassego, jest może intuicyjnie oczywisty, ale mimo to podamy jego dowód, opierając się na własnościach acyklicznych grafów skierowanych.
\end{changemargin}

\textbf{Twierdzenie}
\begin{changemargin}{160pt}{0pt}
\noindent\fbox{%
    \parbox{5in}{%
        Każdy skończony zbiór częsciowo uporządkowany ma diagram Hassego.
    }%
}

\qquad \textbf{Dowód}. Dla danego zbioru częściowo uporządkowanego $(\mathit{P}, \preceq)$ niech $\mathit{H}$ będzie grafem skierowanym o zbiorze wierzchołków $\mathit{P}$ i w którym od wierchołka $\mathit{x}$ do wierzchołka $\mathit{y}$ biegnie krawędź wtedy i tylko wtedy, gdy $\mathit{x}$ nakrywa $\mathit{y}$. Typowa droga w grafie $\mathit{H}$ ma ciąg wierzchołków $x_{1}x_2...x_{n+1}$, w którym $x_{1}$ nakrywa $x_{2}$, $x_{2}$ nakrywa $x_{3}$ itd., a więc $x_{1} \succ x_{2} \succ ... \succ x_{n+1}$. Na mocy przechodności i antysymetrii, $x_{1} \succ x_{n+1}$; w szczególności $x_{1} \ne x_{n+1}$ i nasza droga nie jest zamknięta. Zatem $\mathit{H}$ jest acyklicznym grafem skierowanym. Pokazaliśmy w paragrafach 7.3 i 8.1, że każdy acykliczny graf skierowany ma etykietowanie uporządkowane. Wybierając dla grafu $\mathit{H}$ takie etykietowanie i robiąc jego rysunek wten sposób, by wierchołki z większymi liczbami były wyżej, otrzymamy diagram Hassego dla $(\mathit{P}, \preceq)$.

\qquad Niektóre nieskończone zbiory częściowo uporządkowane również mają diagramy Hassego. Diagram Hassego zbioru $\mathcal{Z}$ ze zwykłym porządkiem $\lesseq$ składa się z kropek rozmieszczonych w jednakowych odstępach wzdłuż pionowe prostej. Z drugiej strony, ponieważ żadna liczba rzeczywista nie nakrywa żadnej innej w sensie zwykłego porządku $\lesseq$, więc zbiór częsciowo uporządkowany $(\mathcal{R},\lesseq)$ nie ma dragramu Hassego.
\end{changemargin}
\end{document}
